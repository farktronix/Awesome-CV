%-------------------------------------------------------------------------------
%	SECTION TITLE
%-------------------------------------------------------------------------------
\cvsection{Work Experience}


%-------------------------------------------------------------------------------
%	CONTENT
%-------------------------------------------------------------------------------
\begin{cventries}

%---------------------------------------------------------
  \cventry
    {Senior Software Engineer} % Job title
    {Apple, Inc} % Organization
    {Cupertino, California} % Location
    {July 2005 - May 2018} % Date(s)
    {
    \vspace{-2.0mm}
        	\begin{cvitems}
         	\item {Contributed to twelve major iOS software releases that have been installed on over a billion devices.}
         	\item {One of the most prolific bug writers at Apple with over 10,000 bug reports written.}
         	%\item {Developed a number of internal tools that are still widely used across the company.}
		\item {Presented three times on stage at the Apple Worldwide Developers Conference (\colorlink{https://developer.apple.com/videos/play/wwdc2014/231/}{2014 - Session 231}, \colorlink{https://developer.apple.com/videos/play/wwdc2016/226/}{2016 - Session 226}, \colorlink{https://asciiwwdc.com/2010/sessions/129?q=local\%20notifications}{2010 - Session 129}).}
          \end{cvitems}
        %\cvsubsection{Projects}
	\begin{cvsubentries}
        		\cvsubentry{\faCloud}{\href{https://developer.apple.com/icloud/cloudkit/}{CloudKit}}{2012 - 2018}
       		{CloudKit is an API that enables applications to easily access and store data on the iCloud servers. \newline
		CloudKit serves as the backend for iCloud Photo Library, iCloud Drive, Notes, and many App Store apps. As of 2016 there were 782 million iCloud users.
		\vspace{2.0mm}
		\begin{cvitems}
         		\item {Co-creator of the \colorlink{https://developer.apple.com/documentation/cloudkit}{CloudKit API} and code base.}
			\item {Maintained an excellent relationship with the CloudKit server team to design and implement features.}
         		\item {Worked closely with many teams inside and outside Apple to support their use of the CloudKit API.}
         		\item {Added sharing features in 2016, which required extensive design work with server and user teams.}
			\item {Focused on client-side encryption features. Ensuring user privacy and security was a top goal for the project.}
         	 \end{cvitems}
        		}
		\cvsubentry{\faCalendar}{Calendar Sync}{2009 - 2015}
       		{%CalDAV is an \href{https://tools.ietf.org/html/rfc4791}{open standard} for synchronizing calendar items using the WebDAV protocol.
		\vspace{-2.0mm}
		\begin{cvitems}
         		\item {Created a calendar synchronization daemon for the iOS Calendar app using the open \href{https://tools.ietf.org/html/rfc4791}{CalDAV} standard.}
			\item {Added scheduling and sharing support on iOS.} 
         		\item {Participated in the \href{https://www.calconnect.org}{CalConnect} working group and attended numerous interop meetings to ensure iOS compatibility with other CalDAV clients and servers.}
			\item {Worked closely with server engineers both within Apple and from other companies.}
         	 \end{cvitems}
        		}
		\cvsubentry{\faRepeat}{iPhone Backup}{2008 - 2011}
       		{
		\vspace{-2.0mm}
		\begin{cvitems}
         		\item {Wrote a service to back up user data from iPhones to iTunes.}
			\item {Added support for encrypted backups and backing up App Store apps.}
         	 \end{cvitems}
        		}
		\cvsubentry{\faTablet}{iPhone}{2006 - 2011}
       		{
		\vspace{-2.0mm}
		\begin{cvitems}
			\item {Worked on the initial creation of the iPhone, contributing code fixes, tools, and bug reports across the entire project.}
         		\item {Developed synchronization code for Contacts, Calendars, Bookmarks, Mail Accounts, and Notes to iPhones via iTunes.}
         	 \end{cvitems}
        		}
		\cvsubentry{\faRefresh}{Sync Services}{2005 - 2006}
       		{Sync Services is the OS X API for synchronizing calendars, contacts, and other data with phones and .Mac.
		\vspace{2.0mm}
		\begin{cvitems}
         		\item {Wrote and maintained unit tests, screened bugs, and contributed code fixes.}
         	 \end{cvitems}
        		}
      	\end{cvsubentries}
    }
\end{cventries}