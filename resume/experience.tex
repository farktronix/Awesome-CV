%-------------------------------------------------------------------------------
%	SECTION TITLE
%-------------------------------------------------------------------------------
\cvsection{Work Experience}


%-------------------------------------------------------------------------------
%	CONTENT
%-------------------------------------------------------------------------------
\begin{cventries}

%---------------------------------------------------------
  \cventry
    {Senior Software Engineer} % Job title
    {Apple, Inc} % Organization
    {Cupertino, California} % Location
    {July 2005 - May 2018} % Date(s)
    {
    \vspace{-2.0mm}
        	\begin{cvitems}
         	\item {Contributed to twelve major iOS software releases.}
         	\item {One of the most prolific bug filers at Apple with over 10,000 bugs filed.}
         	\item {Developed a number of internal tools that were widely used across the company.}
		\item {Presented three times on stage at the Apple Worldwide Developers Conference.}
          \end{cvitems}
        \cvsubsection{Projects}
	\begin{cvsubentries}
        		\cvsubentry{\faCloud}{CloudKit}{2012 - 2018}
       		{CloudKit is an API that enables applications to easily access and store data on the iCloud servers. \newline
		CloudKit serves as the backend for iCloud Photo Library, iCloud Drive, Notes, and many App Store apps. As of 2016 there were 782 million iCloud users.
		\vspace{2.0mm}
		\begin{cvitems}
         		\item {Designed and implemented the CloudKit API from scratch.}
         		\item {Worked closely with many teams inside and outside Apple to support their CloudKit adoption.}
         		\item {Added sharing features in 2016, which required extensive design work with server and user teams.}
			\item {My work focused on client-side encryption features. Ensuring user privacy was a top goal for the project.}
         	 \end{cvitems}
        		}
		\cvsubentry{\faCalendar}{CalDAV}{2009 - 2015}
       		{CalDAV is an \href{https://tools.ietf.org/html/rfc4791}{open standard} for synchronizing calendar items using the WebDAV protocol.
		\vspace{2.0mm}
		\begin{cvitems}
         		\item {Developed initial CalDAV calendar synchronization support for the iOS Calendar app}
			\item {Added scheduling and sharing support on iOS.} 
         		\item {Participated in the CalConnect working group and attended numerous interop meetings to ensure iOS compatibility with other CalDAV clients and servers.}
			\item {Worked closely with server engineers both from with Apple and from other companies.}
         	 \end{cvitems}
        		}
		\cvsubentry{\faRepeat}{iPhone Backup}{2008 - 2011}
       		{
		\vspace{-2.0mm}
		\begin{cvitems}
         		\item {Created code to back up user data from iPhones to iTunes.}
			\item {Added support for encrypted backups and backing up App Store apps.}
         	 \end{cvitems}
        		}
		\cvsubentry{\faTablet}{iPhone}{2006 - 2011}
       		{
		\vspace{-2.0mm}
		\begin{cvitems}
			\item {Worked on the initial development of the iPhone, contributing code fixes, tools, and bug reports across the entire project.}
         		\item {Developed synchronization code for Contacts, Calendars, Bookmarks, Mail Accounts, and Notes to iPhones via iTunes.}
         	 \end{cvitems}
        		}
		\cvsubentry{\faRefresh}{Sync Services}{2005 - 2006}
       		{Sync Services is the OS X API for synchronizing calendars, contacts, and other data with phones and .Mac.
		\vspace{2.0mm}
		\begin{cvitems}
         		\item {Wrote and maintained unit tests, screened bugs, and contributed code fixes.}
         	 \end{cvitems}
        		}
      	\end{cvsubentries}
    }
\end{cventries}